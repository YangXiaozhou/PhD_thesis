\chapter{Literature Review}
\label{ch:literature}
\vspace{2em}

The state-of-the-art research works in power system real-time line outage detection and identification using PMU data are reviewed in this section. 

\section{Outage Detection} % (fold)
\label{sub:outage_detection}

Research that deals with detecting a line outage as fast as possible after it happened can be summarized from two aspects: the approach and the type of system dynamics considered. Outage detection research by their approach is first reviewed. Then, from the perspective of system dynamics, current work is examined for further research gaps. Finally, research work that focuses on outage line identification is reviewed.

\subsection{By Approach} % (fold)
\label{ssub:by_approach}

Most works of line outage detection using PMU data can be classified by the two approaches taken. One is a data-driven approach where no or very little physical knowledge about the system is required~\cite{Xie2014, Rafferty2016, Hosur2019}. On the other hand, many take a hybrid approach where first-principle models are incorporated with data-driven methods~\cite{Jamei2016, Jamei2017a, ardakanian2017event, Ardakanian2019a, Tate2008, tate2009double, dai2020line,Chen2016, Rovatsos2017}. 

\subsubsection{Data-driven Approach}
% Approximation error based on PCA of frequency and voltage data
Authors in \cite{Xie2014} focus on analyzing the underlying dimensionality of the raw PMU data obtained. Using projection by principal component analysis (PCA), they then build a lower-dimensional representation of observable bus states from available PMU data under an outage-free condition~\cite{Xie2014}. An event indicator is constructed based on the reconstruction error between the representation and the actual measurements. 
% Control charts based on PCA of frequency difference data
Similarly, using a moving-window PCA on normal condition system-wide frequency data, Rafferty $\etal$ design a Hotelling’s $T^2$ control chart to detect and classify multiple types of abnormal frequency events~\cite{Rafferty2016}. While many works focus on the detection of a single event, this work considers detecting cascading failures where a series of disruptive events happen in a short time window. They investigated the ability of the proposed method to identify both simulated and real cascading events such as generation dip followed by a loss of load, multiple islanding from line tripping events, and short circuit event followed by an islanding event.

% Null space approximation error monitoring based on system identification
Hosur and Duan proposed constructing an observation matrix based on the frequency difference between buses under a normal condition by modeling the network as a linear time-invariant (LTI) system~\cite{Hosur2019}. An alarm is raised whenever the underlying null space of the observation matrix changes due to different types of events, such as topology change or forced oscillation. The method is not limited to line outages but requires a window of samples to reflect a null space change. 

Without a model for the power system based on physical laws, these data-driven schemes are flexible enough to detect both outages and other abnormal events. However, they often face difficulties when the events have a low signal-to-noise ratio, e.g., outages with mild phase angle disturbances. On the other hand, the hybrid approach augments PMU data with physical system information to improve the detection performance under such conditions.

\subsubsection{Hybrid Approach}
% Using Ohm's law, monitoring by rank-1 approximation error of I,V correlation matrix
One group of work that takes a hybrid approach to outage detection makes use of the Ohm’s law, which relates voltage, current, and line admittance. Jamei $\etal$ show that the correlation matrix between voltage and current measurements of a pair of buses has rank one under normal conditions~\cite{Jamei2016,Jamei2017a}. A cumulative sum (CUSUM) control chart is then set up to monitor the approximation error based on the rank-1 assumption. An outage alarm is raised once the error deviates significantly from zero. The authors devised both a local and central rule for abnormal event detection. The formulation also considers the unique condition of unbalanced phases in a distribution grid. However, currents and voltages at both ends of the line are assumed to be known. Also, the focus of the detection scheme is on deriving the signal without a systematic approach to designing a monitoring scheme that balances detection speed and false alarm rate. 
% Using Ohm's law, monitoring by admittance matrix recovery through lasso
Also based on Ohm’s law, Ardakanian $\etal$ focus on monitoring the discrepancy between measured and computed quasi-steady-state current phasors using the recovered admittance matrix~\cite{ardakanian2017event,Ardakanian2019a}. The post-outage system admittance matrix is recovered from PMU data by solving an inverse power flow and system identification problem. Since the method is based on the admittance matrix, it has the potential to be extended to detecting other disruptive events which induce a change in the matrix.

A separate line of work makes use of the direct current (DC) power flow model, which describes the relationship between substation power injections, voltage, and line admittance~\cite{Chen2016,Babakmehr2016}. In general, they focus on monitoring the pre- and post-outage steady-state voltage phase angle difference. Outage detection is then formulated as a quickest change detection problem. The detection is done by sequential likelihood ratio testing. 
This line of work does not require all buses to be monitored by a PMU. However, the steady-state approximation would not be sufficient at describing the actual system dynamic behavior following an outage. 
% subsection by_approach (end)

Therefore, existing methods do not consider the transient dynamics following an outage or require a full PMU deployment. There is minimal work on detection schemes that consider unobserved buses and exploit the system's transient response to an outage for faster detection.

\subsection{By System Dynamics} % (fold)
\label{ssub:by_dynamics}

In addition to the usage of physical system models, most state-of-the-art works on outage detection can also be categorized by the type of power system transient dynamics, the evolution of system states following a disturbance, considered in their problem formulation. A review from this perspective allows new research direction to be discovered.

\subsubsection{Steady State}
The first group models power systems based on the quasi-steady-state assumption where no transient dynamics are considered~\cite{Tate2008, tate2009double, babakmehr2015application,Chen2016, ardakanian2017event,Ardakanian2019a}. Their detection methods assume that the system is in a quasi-steady-state both before and after the outage. Under this assumption, the DC power flow model, which simplifies many details of the system, is usually used as the starting point for detection scheme design. However, transient dynamics can often last up to several seconds and are non-negligible in real-time operations~\cite{Glover2012}. Therefore, this approach may not be adequate at describing the actual system behavior. 

\subsubsection{Approximate Dynamics}
The second group relaxes the quasi-steady-state assumption and attempts to account for the post-outage transient dynamics. That means the voltage and current profiles of buses and transmission lines are assumed to be time-variant rather than static, especially after an outage. 

Building on the sequential likelihood ratio testing approach in~\cite{Chen2016}, Rovatsos $\etal$ modeled the dynamic evolution of post-outage voltage phase angles using a series of time-dependent participation factor matrices. The matrices quantify the impact of an outage on system bus voltage phase angles based on topology and current operating states~\cite{Rovatsos2017}. However, it is not clear how each of the matrices could be obtained given a new system design and further investigation is needed. Similarly, using voltage angles, a generalized likelihood ratio-based detection scheme is developed using the AC power flow model in Chapter \ref{ch:detection_using_approximate_dynamics} of this thesis. 

Separately, monitoring is also done on low-dimensional subspaces derived from PMU measurements that capture post-outage transient dynamics. Methods such as principal component analysis (PCA)~\cite{Xie2014}, moving-window PCA~\cite{Rafferty2016}, and hidden Markov model (HMM)~\cite{Huang2016b} are used. Jamei $\etal$ proposed to monitor the correlation matrix obtained from adjacent bus voltage and current phasors~\cite{Jamei2017a}; Hosur and Duan proposed to monitor that of the observation matrix obtained during outage-free operation~\cite{Hosur2019}. 

Overall, these methods can capture post-outage system dynamics in a more realistic way than those assuming quasi-steady-state operations. However, all of them rely on system algebraic variables, e.g., bus voltage and current phasors to model the dynamics. System transient dynamics after an outage come from power generators \cite{Glover2012}. Therefore, power generator state variables can better characterize the system's transient response to the power imbalance created by the outage.

\subsubsection{Generator Dynamics}
Based on the generator dynamics insight, the third group models the power system as a dynamical system, utilizing both the measurable algebraic variables and hidden generator state variables. The hope is that the addition of generator information could help better identify line outages. Using the swing equation, a second-order differential equation describing the dynamics of generators, Pan $\etal$ formulated outage diagnosis as a sparse recovery problem solved by an optimization algorithm~\cite{Pan2015a}. Similarly, a visual observer network is constructed using the swing equation to monitor line admittance changes by a parameter identification method~\cite{Yang2016b}. However, both of these works focus on the outage diagnosis problem, i.e., outage line identification and post-outage parameter estimation. However, a systematic detection scheme is the prerequisite for such tasks and needs to be developed.
% subsection by_dynamics (end)

Therefore, there is limited work on systematic line outage detection  considering generator dynamics in a partially observed network. No work has been done to integrate generator information and power flow algebraic variable information for faster and more reliable outage detection.


% section outage_detection (end)

\section{Outage Identification} % (fold)
\label{sub:outage_identification}
Early detection of line outage may not lead to better system protection if the location of the outage line(s) are not known. Research on outage lines identification has also thrived in the past twenty years, mainly taking three directions. Research work that focuses on identifying outage lines by estimating post-outage line parameters is reviewed first. Then, those with a machine learning approach and expected phase angle change approach are reviewed.

% Line parameter change diagnostics
\subsection{Parameter Recovery Approach}
Line outage identification, in general, can be formulated as a parameter change identification problem. An outage of line l corresponds to the change when the line admittance of $\ell$ drops to zero. When post-outage system parameters are estimated, a comparison with pre-outage baseline information could reveal outage locations. Some attempted to solve this problem by recovering changed line parameters. Yu $\etal$ use the power flow equations to formulate line parameters as unknown regression coefficients and recover them via the total least squares method in~\cite{Yu2018} and considering changing baseline conditions~\cite{Yu2019}. Another approach focuses on the system admittance matrix, which depends on line-bus connection information, and recovers the matrix elements via matrix decomposition and adaptive lasso~\cite{Babakmehr2016,Ardakanian2019a}. 

These methods can both locate multiple line outages and recover post-outage system parameters. However, they either require a full PMU deployment or smart meter measurements, e.g., power injection and voltage measurements for the power flow approach, current and voltage measurements for the admittance matrix approach at all buses.

% Machine learning-based approach
\subsection{Machine Learning Approach}
However, it is more realistic to assume only part of the system is observable by PMUs or smart meters, as mentioned in the challenge section. Under this assumption, researchers mainly seek to solve line outage identification problems, i.e., locate outage lines. Two main approaches are taken. The first, machine learning-based approach, is reviewed here. This line of work leverages easily accessible simulated power system outage data, extracts valuable information from them, and trains an outage classifier in a supervised learning setting. The classifier then identifies the most likely outage locations when given a new set of system data. Classical supervised learning techniques such as multinomial logistic regression classifiers are used~\cite{Garcia2016,Kim2018}. Recently, convolutional and Bayesian neural networks, a family of versatile machine learning techniques, are proposed to identify tripped lines~\cite{Li2019a,Zhao2020}. 

The machine learning-based approach exploits the ability of these algorithms to learn an excellent representation of line outages. They are powerful at locating multiple line outages with limited PMUs. However, their performance depends on generalizable and usually massive training data.

% Expected angle change approach
\subsection{Expected Angle Change Approach}
The second approach that does not demand a full PMU deployment on the power system is the expected voltage phase angle change formulation. Different from the machine learning approach, this line of work does not depend on training data either. Instead, well-known physical laws governing power systems are utilized to construct a ``dictionary'' of how voltage phase angles might change following any outage. That knowledge is then used to decide where outages might have happened after collecting post-outage system data. This dictionary of patterns of angle change is usually based on either the DC power flow model~\cite{Tate2008,tate2009double,Zhu2012,Emami2013,Chen2014,Wu2015} or the AC model~\cite{Costilla-Enriquez2019}. Outage line identification is then formulated into an unknown sparse vector recovery problem or a pattern-matching problem. 
For example, the outage status of all lines is formulated as an unknown vector recovered by optimization methods such as orthogonal matching pursuit, mixed-integer programming, cross-entropy optimization, and matrix decomposition~\cite{Zhu2012, Emami2013,Chen2014,Wu2015}. Also, Tate and Overbye and Enriquez $\etal$ used the correlation between measurement data and expected phase angle data to identify the most likely outage locations~\cite{Tate2008,Costilla-Enriquez2019}. 

However, the usage of the DC power flow model potentially creates system representations with fidelity inadequate for accurate single- and multiple-line outage identification. All but Enriquez $\etal$ consider the AC power flow model. However, their approach requires both voltage and current information. Overall, despite demonstrated effectiveness by the above works, the identification performance degrades significantly when a limited number of PMUs are available or when multiple-line outages are considered. 

% section outage_identification (end)

% Current gaps
Therefore, given an outage detection timestamp, multiple-line outage identification problem remains difficult when (1) massive generalizable training data is not available or feasible, (2) only a portion of the system buses are equipped with PMUs, (3) only bus voltage phasor information is used.  
