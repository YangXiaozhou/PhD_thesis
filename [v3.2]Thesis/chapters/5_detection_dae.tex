\SetPicSubDir{4_detection_dae}
\SetExpSubDir{4_detection_dae}


\chapter{Outage Detection Using Generator and Load Bus Dynamics} % (fold)
\label{ch:detection_using_generator_dynamics}


\section{Introduction} % (fold)
\label{ch4:sec:introduction}

A unified detection framework that monitors both generator dynamics and load bus power changes using a limited number of PMUs is proposed in this chapter. System transient dynamics are tracked through nonlinear state estimation via a particle filter (PF). A statistical change detection scheme is constructed by monitoring the PF-predicted system output's compatibility with the expected normal-condition measurement. When an outage happens, a significant reduction in the compatibility triggers an alarm, detecting the outage in the quickest time possible. 

The rest of this chapter is organized as follows. A unified outage detection scheme based on nonlinear power system dynamics is formulated in Section \ref{ch4:sec:power_model} and Section \ref{ch4:sec:detection_scheme}. Section \ref{ch4:sec:state_estimation} then describes the PF-based online state estimation necessary for tracking generator dynamics. The proposed scheme's effectiveness and advantages are presented in Section \ref{ch4:sec:results} using simulation studies. Section \ref{ch4:sec:conclusion} is the conclusion.

% section introduction (end)

\section{Power System State-Space Modelling}
\label{ch4:sec:power_model}
%%%%%%%%%%%%%%% Dynamic equations %%%%%%%%%%%%%%%
This section describes a power system model that captures both the generator dynamics and network power flow information in a unified framework. Consider a power system with $M$ generator buses where $\mathcal{N}_g = \{1, \dots, M\}$, $N-M$ load buses where $\mathcal{N}_l = \{M+1, \dots, N\}$, and $L$ transmission lines where $\mathcal{L} = \{1, \dots, L\}$. 
The power system is a hybrid dynamic system described by a differential-algebraic model. The second-order generator model, also known as the swing equation \cite{Kundur1994}, is used in this work\footnote{Although the swing equation is used here to model generator rotor dynamics, high-order and more complex models, such as the two-axis model \cite{sauer2017power}, can be used. The detection scheme proposed in this work can be developed similarly.}.
For every generator bus $i \in \mathcal{N}_g$, their states are modeled as the differential variables, i.e., $\boldsymbol{X}= [\delta, \omega]^T$ where $\delta$ is the rotor angular position in radians with respect to a synchronously rotating reference, and $\omega$ is the rotor angular velocity in radians/second. The differential equations governing their dynamics are
\begin{subequations}
\label{ch4:eqn:de_continuous}
\begin{align}
\dot{\delta}_{i} &=\omega_s\left(\omega_{i}-1\right) \,, \label{ch4:eqn:transition_function_delta}\\
M_i \dot{\omega}_{i} &=\text{P}_{m, i}-\hat{\text{P}}_{g, i}-D_i\left(\omega_{i}-1\right) \,, \label{ch4:eqn:transition_function_omega}
\end{align}
\end{subequations}
where $\dot{\delta}_{i}$ is the derivative of $\delta_i$ with respect to $t$. $\omega_s$ is the synchronous rotor angular velocity such that $\omega_s= 2 \pi f_{0}$ where $f_0$ is the known synchronous frequency. $\text{P}_{m,i}, M_i$, and $D_i$ denote the mechanical power input, the inertia constant, and the damping factor, respectively. They are assumed known and constant for the duration of this study. The inputs for the model are the generated active power, i.e., ${u}= \text{P}_{g}$. Under the classical generator model assumptions, the synchronous machine is represented by a constant internal voltage $\text{E}\angle\delta$ behind its direct axis transient reactance $\text{X}_{{d}}^{\prime}$ \cite{Kundur1994}. Therefore, the active power at generator $i$ is
\begin{equation}
\label{ch4:eqn:ae_continuous}
\text{P}_{g, i} = \frac{\text{E}_i\text{V}_i}{\text{X}_{d, i}^{'}}\sin (\delta_i - \theta_i)\,,
\end{equation}
where $\theta$ is the generator bus nodal voltage phase angle. The transient reactance is assumed known and constant, whereas a method will be presented later to adaptively infer the parameter $\text{E}$ with online data. Also, denote 
$$
\hat{\text{P}}_{g, i} = \text{P}_{g, i} + \epsilon_i \,,
$$
where $\epsilon$ is assumed to be a zero-mean Gaussian process with a known variance representing the random fluctuations in electricity load on the bus as well as process noise. 

%%%%%%%%%%%%%%% Measurements and power flow balance %%%%%%%%%%%%%%%
The outputs of the system model are nodal voltage magnitudes and phase angles which PMUs can measure. More importantly, the algebraic output and generator states have to satisfy an active power balance constraint. The constraint stipulates that the net active power at a bus is the difference between the active power supplied to it by the generator and the load consumed, i.e.,
\begin{equation}
\label{ch4:eqn:power_balance}
\text{P}_i = \text{P}_{g, i} - \text{P}_{l, i} \,,
\end{equation}
for $i = 1, \dots, N$, subject to a random demand fluctuation $\epsilon_i$ as mentioned above. $\text{P}_{l, i}$ is the load on bus $i$, $\text{P}_{i}$ is the nodal net active power and 
\begin{equation}
\label{ch4:eqn:ac_pf}
\text{P}_{i} = \text{V}_i \sum_{j=1}^{N} \text{V}_j \text{Y}_{ij} \cos (\theta_i - \theta_j - \alpha_{ij}) \,,
\end{equation} following the AC power flow equation of (\ref{ch2:eqn:AC_power_flow_P}) where $\text{Y}_{ij}e^{j\alpha_{ij}}$ are elements of the bus admittance matrix. Note that for load buses $\text{P}_{g, i} = 0, i \in \mathcal{N}_l$ in (\ref{ch4:eqn:power_balance}). The total active power generated and load demand of the network are assumed to be balanced as well. This relationship will be the basis for the unified outage detection scheme described in the next section. 


%%%%%%%%%%%%%%% Discrete system and measurement equations %%%%%%%%%%%%%%%
Define the discrete counterparts of the system model via a first-order difference discretization by Euler's formula, i.e., let $\delta_{k+1} = \delta_{t_{k+1}}$ for $k = 1, 2, \dots$, and 
$$
\dot{\delta}_{t_{k+1}} \approx \frac{\delta_{k+1}-\delta_{k}}{\Delta t} \,.
$$
For PMU devices with a sampling frequency of 30 Hz, $\Delta t = t_{k+1} - t_{k} = 1/30$ s. Thus, the continuous system of a generator bus $i$ can be approximated by
\begin{equation}
\label{ch4:eqn:de_discrete}
\boldsymbol{X}_{i,k+1} = 
\left[
\begin{array}{c}
\delta_{i,k+1} \\
\omega_{i,k+1}
\end{array}
\right] =
\left[
\begin{array}{c}
\delta_{i,k} + \Delta t \omega_s\left(\omega_{i,k}-1\right) \\
\omega_{i,k} + \frac{\Delta t}{M_i}q_{i, k} - \epsilon_{k}\,
\end{array}
\right]  
\end{equation}
where 
$$
q_{i, k} = \text{P}_{m, i}-\text{P}_{g, i,k}-D_i\left(\omega_{i,k}-1\right)
$$
for notational brevity, and 
\begin{equation}
\label{ch4:eqn:p_g_discrete}
\text{P}_{g,i,k} = \frac{\text{E}_i\text{V}_{i, k}}{\text{X}_{d, i}^{'}}\sin (\delta_{i, k} - \theta_{i,k}) \,.
\end{equation}

Taking a derivative with respect to time $t$ on both sides
of (\ref{ch4:eqn:power_balance}) and rearranging the terms, then 
\begin{equation}
    \frac{\partial \text{P}_{l, i}}{\partial t} = \frac{\partial \text{P}_{g, i}}{\partial t} -  \frac{\partial \text{P}}{\partial t} \,,
\end{equation}
relating the changes in bus load to the changes in active power generated and transferred from the bus. The discretized relationship is then
\begin{equation}
\label{ch4:eqn:ae_discrete}
\Delta\text{P}_{l, i, k} = \Delta\text{P}_{g, i, k} -  \Delta\text{P}_{i, k}\,,
\end{equation}
where 
$$
\Delta\text{P}_{l, i, k} = \text{P}_{l, i, k} - \text{P}_{l, i, k-1} \,,
$$
and similarly for the other two terms. Writing the whole system in vector form, we also define the N-dimensional output variable
\begin{equation}
\label{ch4:eqn:ae_discrete}
\boldsymbol{Y}_{k} = \Delta\textbf{P}_{k} =
\Delta\textbf{P}_{g, k}  - \Delta\textbf{P}_{l, k}\,,
\end{equation}
where $\Delta\textbf{P}_{l, k}$ represents the random load fluctuations. $\Delta\textbf{P}_{l, k}$ is assumed to be a zero-mean Gaussian variable with covariance $\sigma^2\mathbf{I}$. Note that the entries corresponding to load buses in $\Delta\textbf{P}_{g, k}$ are all zero. 

% Why is it a unified framework and how are others related
Through (\ref{ch4:eqn:ae_discrete}), the active power changes in both generator and load buses can be monitored. In comparison, detection schemes developed in previous works focus on monitoring changes in net active power, $\Delta\textbf{P}$, through direct current (DC), e.g., \cite{Chen2016}, or AC, e.g., \cite{yang2020control}, power flow equations. Their formulations can be considered as special cases of the proposed unified framework when no generator information is available, e.g., no PMUs are installed on generator buses. However, as shown in simulation studies, having generator power output information helps to detect certain outages when net active power changes are not significant enough to trigger an alarm.

% Power system dae in state-space model framework
(\ref{ch4:eqn:de_discrete})-(\ref{ch4:eqn:ae_discrete}) define a state-space model (SSM) for the power system that could be summarized in the general form below:
\begin{subequations}
\label{ch4:eqn:general_ssm}
\begin{align}
\boldsymbol{X}_{k+1} &= a(\boldsymbol{X}_{k}, \boldsymbol{u}_{k}, \boldsymbol{\epsilon}_{k}) \, \rightarrow f(\boldsymbol{X}_{k+1}|\boldsymbol{X}_{k}) \\ 
\boldsymbol{Y}_{k} &= b(\boldsymbol{X}_{k}, \boldsymbol{u}_{k}, \boldsymbol{\eta}_{k}) \, \rightarrow g(\boldsymbol{Y}_{k} | \boldsymbol{X}_{k})
\end{align}
\end{subequations}
The dynamics of the unobservable generator states $\boldsymbol{X}$ are governed by the state transition function $a(\cdot)$ as in (\ref{ch4:eqn:de_discrete}). The output $\boldsymbol{Y}$, computable from PMU measurements, is governed by the output function $b(\cdot)$ as in (\ref{ch4:eqn:ae_discrete}). Since $b(\cdot)$ is a nonlinear function of the system states, the power system is a nonlinear dynamic system.

As the transition process is stochastic due to random load fluctuations and measurement errors, the states and outputs can be expressed in a probabilistic way. In particular, we denote the state transition density and output density as 
$$
f(\boldsymbol{X}_{k+1}|\boldsymbol{X}_{k}=\boldsymbol{x}_{k})
$$ 
and 
$$
g(\boldsymbol{Y}_{k} |\boldsymbol{X}_{k}=\boldsymbol{x}_{k}) \,,
$$
respectively, where $f(\cdot)$ and $g(\cdot)$ are probability density functions (PDFs). These two densities play important roles in the particle filter-based state estimaton described later in the chapter. An important consequence of the SSM is the conditional independence of the states and output due to the Markovian structure. In particular, given $\boldsymbol{X}_{k}$, $\boldsymbol{X}_{k+1}$ is independent of all other previous states; similarly given $\boldsymbol{X}_{k}$, $\boldsymbol{Y}_{k}$ is independent of all other previous states.


\section{EWMA-Based Outage Detection Scheme}
\label{ch4:sec:detection_scheme}
% Statistical basis for outage detection
A system-wide detection scheme that utilizes the output of the SSM detailed in the previous section is described here. Under an outage-free scenario, the active power generated, transmitted, and consumed in the network are expected to be balanced with only small random load demand fluctuations. Therefore, $\Delta\textbf{P}_{l, k}$, which represents the instantaneous changes in power demand, is assumed to be normally distributed with mean zero under the baseline condition:
\begin{equation}
\label{ch4:eqn:normal_distribution}
\Delta\textbf{P}_{l, k} = 
\Delta\textbf{P}_{g, k} - \boldsymbol{Y}_{k} \sim N(\mathbf{0}, \sigma^2\mathbf{I}) \,.
\end{equation}
After a line outage, there are two ways that the above relationship will be violated. First, the outage-induced topology change means that the line admittance of the tripped line becomes effectively zero; the bus admittance matrix thus changes to reflect the post-outage system topology. Therefore, the outage-free AC power flow equation (\ref{ch4:eqn:ac_pf}) used to compute the net active power is no longer valid. Thus $\boldsymbol{Y}_{k}$ in (\ref{ch4:eqn:normal_distribution}) does not represent the actual net active power changes anymore. Second, the outage triggers a period of transient re-balancing in the system where the generators respond to the sudden power imbalance. The immediately affected buses also experience an abrupt change in the net active power due to the outage. As a combination of these effects, the relationship (\ref{ch4:eqn:normal_distribution}) will be violated. For example, using data simulated from the IEEE 39-bus test system, Fig. \ref{ch4:fig:signals_comparison} shows the signals from a normal system and that with an outage at the third second. 
\begin{figure}[!t]
\centering
\includegraphics[width=1\linewidth]{\Pic{pdf}{signals_comparison}}
\caption{\textit{Comparison of the residual signals with no outage and with line 12 outage. A subset of residual signals significantly deviated from the normal mean level and exhibited strong non-Gaussian oscillations after the outage.}}
\label{ch4:fig:signals_comparison}
\end{figure}

% Statistical change detection via MEWMA
Therefore, the early outage detection problem is formulated as a multivariate process monitoring problem. The multivariate residual signal's significant deviation from the expected distribution indicates an abnormal event, e.g., an outage. For its robustness to non-Gaussian data, superior performance on small to median shifts, and easy of implementation, the multivariate exponentially weighted moving average (MEWMA) control chart, initially developed by \cite{lowry1992multivariate}, is adopted for the detection task. The MEWMA control chart uses an intermediate quantity, $\boldsymbol{Z}_k$, that captures both the current and past signal information from the system. The quantity can be shown to be the weighted average of all past signals with geometrically declining weights. In particular, it can be constructed with the latest SSM output, $\boldsymbol{Y}_k$, as follows:
\begin{equation}
\label{ch4:eqn:ewma_z}
\boldsymbol{Z}_k = \lambda (\Delta\textbf{P}_{g, k} - \boldsymbol{Y}_{k}) + (1 - \lambda) \boldsymbol{Z}_{k-1} \,,
\end{equation}
where $\lambda$ is a pre-defined smoothing parameter that controls the extent of the reliance we would like to put on past information. Also, $0 \le \lambda \le 1$ and $\boldsymbol{Z}_0 = \mathbf{0}$. The statistic under monitoring is then constructed similar to that of a Hotelling's $T^2$ statistic:
\begin{equation}
\label{ch4:eqn:ewma_T}
T^2_k = \boldsymbol{Z}_k^T\Sigma_{\boldsymbol{Z}_k}^{-1}\boldsymbol{Z}_k \,,
\end{equation}
where the covariance matrix is 
$$
\Sigma_{\boldsymbol{Z}_k} = \frac{\lambda}{2 - \lambda}\left[1-(1-\lambda)^{2k}\right]\sigma^2\mathbf{I} \,.
$$
An outage alarm is then triggered when the monitoring statistic crosses a pre-determined threshold, $H$, chosen to satisfy a certain false alarm rate requirement:
\begin{equation}
\label{ch4:eqn:control_chart}
D = \inf\lbrace k\ge1 : T^2_k \ge H \rbrace \,.
\end{equation}
$D$ is the stopping time of the proposed outage detection scheme. Suppose the onset time of the outage is denoted by $t_o$, then the difference between the stopping time of the control chart and the onset time is the detection delay, i.e., $D - t_o$. One way of judging the effectiveness of different detection schemes is by comparing their detection delays against various line outage events. An ideal detection scheme is therefore able to detect an outage immediately after it happened, i.e., $D - t_o = 0$. 

For MEWMA control charts, it is possible to specify a requirement on the false alarm rate through the selection of $\lambda$ and $H$. One way to specify the detection scheme's false alarm rate is through the so-called average run length under zero state ($ARL_0$), i.e., the average number of signals collected before the above detection threshold is reached under an outage-free scenario. Larger $ARL_0$s correspond to more lenient false alarm requirements, but possibly longer detection delays. It is also known that small $\lambda$ values produce control charts more robust against non-Gaussian distributions and have better detection performance for small to medium shifts \cite{montgomery2007introduction}. Given $\lambda$ and a false alarm requirement $ARL_0$, the detection threshold $H$ can be determined by solving an integral equation of Theorem 2 in \cite{rigdon1995integral}\footnote{The equation can be solved using various numerical algorithms or Markov chain approximation, and it be done offline. Interested readers can refer to \cite{knoth2017arl} for a detailed description of the computation procedure required.}. The selection of the parameter values and their impact on the detection scheme will be presented in the simulation study section. 



%%%%%%%%%%%%%%%% State Estimation %%%%%%%%%%%%%%%
\section{Generator State Estimation via Particle Filtering}
\label{ch4:sec:state_estimation}
% Traditional approaches for power system state estimation
% Why particle method
In the previous section, a unified framework of real-time system monitoring utilizing post-outage transient dynamics computed from state and algebraic variables, i.e., active power generated and net active power injection, is described. The premise of the unified framework is the availability of accurate state and algebraic variables data. While algebraic variables can be measured by PMUs, generator states are not directly observable. This section shows how the hidden states could be reliably estimated online using a particle filter.

% Introduce SE problem for SSM
Online state estimation typically involves the inference of the posterior distribution of the hidden states $\boldsymbol{X}_k$ given a collection of output measurements $\boldsymbol{y}_{0:k}$, denoted by $\pi(\boldsymbol{X}_k | \boldsymbol{y}_{0:k})$. This class of marginal state inferences is also known as the filtering problem. When the system can be represented by a linear Gaussian SSM or a finite state-space hidden Markov model, the posterior distribution can be computed in an analytical form using the Kalman technique and Baum-Petrie filter. For systems with nonlinear dynamics and possibly non-Gaussian noises, however, e.g., power systems, the posterior distribution is intractable and cannot be computed in closed form.
% Nonlinear non-Gaussian SSM
Extended and unscented Kalman filters have been extensively studied to address the above problem, e.g., \cite{Zhao2017,Wang2012}. However, these methods' effectiveness becomes questionable when the underlying nonlinearity is substantial or when the posterior distribution is not well-approximated by Gaussian distribution. Instead, PF is increasingly used for this task, e.g., \cite{Cui2015}, as it handles nonlinearity well and accommodates noise of any distribution with an affordable computational cost \cite{Kadirkamanathan2002,cappe2007overview}. PFs belong to the family of sequential Monte Carlo methods where Monte Carlo samples approximate complex posterior distributions, and the distribution information is preserved beyond mean and covariance. 

% Key steps in PF
In particular, PF approximates $\pi(\boldsymbol{X}_k | \boldsymbol{y}_{0:k})$ by samples, called particles, obtained via an importance sampling procedure. Each particle is assigned an importance weight proportional to its likelihood of being sampled from the posterior distribution\footnote{This type of PF is also known as the bootstrap filter first proposed in \cite{Gordon1993}. The idea is to use the state transition density as the importance distribution in the importance sampling step. More sophisticated algorithms, such as the guided and auxiliary particle filter could be implemented in the same detection framework proposed here. However, these algorithms are, in general, more difficult to use and interpret. For details, readers can refer to \cite{doucet2009tutorial}.}. PF proceeds in a recursive prediction-correction framework. Assuming at time $k$, the particles and weights obtained from the previous time step are available as:
$$
\{(\boldsymbol{x}_{k-1}^{i}, w_{k-1}^{i})\}_{1\leq i \leq n},
$$ where $n$ is the number of particles, the posterior distribution at time $k-1$ is approximated by weighted Dirac delta functions as
\begin{equation}
\pi(\boldsymbol{X}_{k-1} | \boldsymbol{y}_{0:k-1}) \approx \sum_{i=1}^{n} w_{k-1}^{i} \cdot \delta(\boldsymbol{X}_{k-1}-\boldsymbol{x}_{k-1}^{i}) \,,
\end{equation} 
where $\delta (\cdot)$ is the Dirac delta function, and the weights are normalized such that $\sum_{i=1}^{n} w_{k-1}^{i} = 1$. The algorithm starts by propagating particles from time $k-1$ to time $k$ through the state transition function in (\ref{ch4:eqn:de_discrete}), i.e., the prediction step. That means, new particles $\{\boldsymbol{x}_k^{i}\}_{1\leq i \leq n}$ are sampled from the state transition density $f(\boldsymbol{X}_k|\boldsymbol{x}_{k-1}^{i})$. The predicted states then have a distribution approximated by 
\begin{equation}
\label{ch4:eqn:prediction}
\pi(\boldsymbol{X}_k | \boldsymbol{Y}_{0: k-1}) \approx \sum_{i=1}^{n} w_{k-1}^{i} \cdot \delta(\boldsymbol{X}_k-\boldsymbol{x}_k^{i}) \,.
\end{equation}
When the new measurement $y_k$ arrives, the above approximation is corrected by updating the particles' weights proportional to their conditional output likelihood to obtain the posterior distribution as 
\begin{equation}
\label{ch4:eqn:particle_approx}
\pi(\boldsymbol{X}_k | \boldsymbol{Y}_{0: k}) \approx \sum_{i=1}^{n} w_k^{i} \cdot \delta(\boldsymbol{X}_k-\boldsymbol{x}_k^{i}) \,,
\end{equation} where 
$$
w_k^{i} \propto  w_{k-1}^{i} \cdot g(\boldsymbol{y}_k | \boldsymbol{x}_k^{i}) \,.
$$
The intuitive interpretation is that the particles are re-weighted based on their compatibility with the actual system measurement. The approximation of the posterior distribution by these particle-weight pairs is consistent as $n \rightarrow +\infty$ at a standard Monte Carlo rate of $\mathcal{O}(n^{-1/2})$ guaranteed by the Central Limit Theorem \cite{doucet2009tutorial}. 

% How to overcome the well-known degeneracy problem
% with systematic resampling
A well-known problem of PF is that the weights will become highly degenerate overtime. In particular, the density approximation will be concentrated on a few particles, and all the other particles carry effectively zero weight. A common way to evaluate the extent of this degeneracy is by using the so-called Effective Sample Size (ESS) criterion \cite{liu2008monte}:
\begin{equation}
\text{ESS}=\left(\sum_{i=1}^{n}\left(w_k^{i}\right)^{2}\right)^{-1} \,.
\end{equation}
In the extreme case where one particle has the weight of 1 and all others of 0, ESS will be 1. On the other hand, ESS is $n$ when every particle has an equal weight of $n^{-1}$.

A resampling move can be used to mitigate the degeneracy problem. The central idea is to duplicate particles with higher weights and remove the others, thus focusing computational efforts on regions of higher probability. The systematic resampling method is used in this case as it usually outperforms other resampling algorithms \cite{doucet2009tutorial}. When ESS falls below a threshold, typically $n/2$, $n$ particles are resampled from the existing ones. The number of offspring, $n_k^{i}$, is assigned to each particle $\boldsymbol{x}_k^i$ such that 
$$
\sum_{i=1}^{n}n_k^{i} = n \,.
$$
The systematic sampling proceeds as follows to select the number of offspring $n_k^{i}$. A random number $U_{1}$ is drawn from the uniform distribution 
$$
\mathcal{U}\left[0, {n}^{-1}\right] \,.
$$
Then a series of ordered numbers are obtained by 
$$
U_{i}=U_{1}+\frac{i-1}{n} \,,
$$
for $i=2, \ldots, n$. $n_k^{i}$ is the number of $U_{i} \in(\sum_{s=1}^{i-1} w_{s}, \sum_{s=1}^{i} w_{s}]$ where $\sum_{s=1}^{0} w_{s} := 0$ by convention. Finally, resampled particles are each assigned an equal weight $n^{-1}$ before a new round of prediction-correction recursion begins. The detailed PF algorithm with the resampling move is summarized in Algorithm \ref{alg:particle_filter}.
% The particle filter algorithm
\begin{algorithm}
\caption{Particle Filter for Generator State Estimation}\label{alg:particle_filter}
\begin{algorithmic}[1]
\Procedure{Initialization}{$n, \pi_0(\boldsymbol{X}), \boldsymbol{y}_0$}
\For{$i = 1, \dots, n$}
\State Sample $\tilde{\boldsymbol{x}}_0^{i} \sim \pi_0(\boldsymbol{X})$.
\State Compute initial importance weight
$
\tilde{w}_0^{i} = g(\boldsymbol{y}_0 | \tilde{\boldsymbol{x}}_0^{i})
$ by output function (\ref{ch4:eqn:ae_discrete}).
\EndFor
\State \textbf{return} $\{(\tilde{\boldsymbol{x}}_{0}^{i}, \tilde{w}_{0}^{i})\}_{1\leq i \leq n}$
\EndProcedure
\Statex
\Procedure{Filtering}{$n, \{(\tilde{\boldsymbol{x}}_{k-1}^{i}, w_{k-1}^{i})\}_{1\leq i \leq n}, \boldsymbol{y}_k$}
\If{$\text{ESS} \le n/2$} \Comment{Systematic resampling}
\State Draw $U_{1} \sim \mathcal{U}\left[0, {n}^{-1}\right]$ and obtain $U_{i}=U_{1}+\frac{i-1}{n}$ for $i=2, \ldots, n$.
\For{$i = 1, \dots, n$}
\State Obtain $n_k^{i}$ as the number of $U_i$ such that
$$
U_i \in \left(\sum_{s=1}^{i-1} w_{s}, \sum_{s=1}^{i} w_{s}\right] \,.
$$ 
\State Select $n$ particle indices $j_i \in \{1, \dots, n\}$ according to $n_k^{i}$.
\State Set $\boldsymbol{x}_{k-1}^{i} = \tilde{\boldsymbol{x}}_{k-1}^{j_i}$, and $w_{k-1}^{i} = 1/n$.
\EndFor
\Else 
\State Set $\boldsymbol{x}_{k-1}^{i} = \tilde{\boldsymbol{x}}_{k-1}^{i}$ for $i=1, \ldots, n$.
\EndIf
\For{$i = 1, \dots, n$}
\State Predict $k_{th}$ system state by sampling particles via state transition function (\ref{ch4:eqn:de_discrete}): \Comment{State prediction}
$$
\tilde{\boldsymbol{x}}_k^{i} \sim f(\boldsymbol{X}_k|\boldsymbol{x}_{k-1}^{i}) \,.
$$ 
\State Update particle weights using current output measurement via output function (\ref{ch4:eqn:ae_discrete}): \Comment{Weight correction}
$$
\tilde{w}_k^{i} = {w}_k^{i} \times g(\boldsymbol{y}_k | \tilde{\boldsymbol{x}}_k^{i}) \,.
$$
\EndFor
\State Normalize weights
$$
w_k^{i} = \frac{\tilde{w}_k^{i}}{\sum_{k=1}^{n}\tilde{w}_k^{k}} \,, \text{for } i = 1, \dots, n \,.
$$
\State \textbf{return} $\{(\tilde{\boldsymbol{x}}_{k}^{i}, w_{k}^{i})\}_{1\leq i \leq n}$
\EndProcedure
\end{algorithmic}
\end{algorithm}



\section{Additional Remarks} % (fold)
\label{ssub:additional_remarks}
\subsection{Limited PMU Deployment}
Many power systems have to work with a limited number of PMUs, i.e., some buses are not equipped with a PMU. The detection scheme proposed here is also applicable in this case since the signal under monitoring, $\boldsymbol{Y}$, can be adjusted to include only buses with PMUs. In particular, $\Delta\textbf{P}_{g}$ can include those generator buses with PMUs. $\Delta\textbf{P}$ can be calculated for load buses with fully observable neighbor buses. The impact of an unobserved neighbor bus on the computation of the bus net active power would be an unknown term, $\text{V}_j \text{Y}_{ij} \cos (\theta_i - \theta_j - \alpha_{ij})$, in the AC power flow equation since the neighbor bus' $\theta_j$ and $\text{V}_j$ are not available. While this impact can be mitigated through a careful selection of the PMU locations, unlike \cite{Chen2016} and \cite{yang2020control}, the proposed detection scheme is effective when most generator buses are monitored, a result corroborated by the simulation studies in this work, e.g., see Fig. \ref{ch4:fig:detection_delay_distribution}. Also, the number of generator buses is typically much smaller than the total number of buses.

\subsection{Unknown System Parameter Estimation}
In this work, it is assumed that the system parameters in the power system SSM are known and static; therefore, the PF's state estimation is reliable. In real-world applications, these parameters may be known but slow-varying due to factors like system degradation. While parameter estimation in a non-linear system is generally a difficult problem and outside the scope of this paper, there is a natural extension from the particle filtering framework that can tackle the problem. An online expectation maximization (EM) algorithm based on the particles can be implemented to learn the parameters as data arrives sequentially in real-time. The EM algorithm is an iterative optimization method that finds the maximum likelihood estimates of the parameters in problems where hidden variables are present \cite{dempster1977maximum}. This basic EM algorithm can be reformulated to perform the estimation online using the so-called sequential Monte Carlo forward smoothing framework when the complete-data density, i.e., $p_\vartheta(\boldsymbol{x}_{0:k}, {\boldsymbol{y}}_{0:k})$ where $\vartheta$ denote the set of unknown parameters, is from the exponential family \cite{yildirim2013online}.


\section{Simulation Study}
\label{ch4:sec:results}
\subsection{Simulation Setting}
The proposed PF-based outage detection scheme is tested on the IEEE 39-bus 10-machine New England system~\cite{athay1979practical}. System transient responses after an outage are simulated using the open-source dynamic simulation platform COSMIC \cite{Song2016}. The simulation results are assumed to be the true generator states, and corrupted measurements are synthesized from the noise-free simulation data. Ten PMUs are assumed to be installed at bus 19, 20, 22, 23, 25, 33, 34, 35, 36, and 37, covering five generator buses and their connected load buses. Their sampling frequency is assumed to be 30 samples per second. Each simulation runs for 10 seconds, and the outage happens at the third second. A line outage is detected if the monitoring statistic crosses the detection threshold by the end of the simulation. The detection thresholds of all schemes presented are selected by satisfying a false alarm constraint of 1 in 30 days. The global constants are $f_0 = 60 $ Hz and $\omega_s = 1.0 $ p.u.. For the SSM, state function noise $\epsilon_k$ are assumed to be uncorrelated and homogeneous with a standard deviation of $0.01\% \cdot \text{P}_{g,k}$ in (\ref{ch4:eqn:de_discrete}). Output function error $\boldsymbol{\eta}_k$ are assumed to follow a zero-mean Gaussian distribution with a standard deviation of 
$1\% \cdot (\text{P}_{g, k}- \text{P}_{k})$ in (\ref{ch4:eqn:ae_discrete}). 


\subsection{Illustrative Outage Detection Example}  
\label{ch4:sec:results:example}
% Typical progression of monitoring statistics
To illustrate the working of the detection scheme, line 11 outage is used as an example. Fig. \ref{ch4:fig:line_18_state_estimation_eg} shows a typical performance of the particle filter used to estimate generator bus states. The rotor angular speed, $\omega$, can be accurately tracked while the rotor angular position, $\delta$, has some biases after the outage. This is acceptable since the focus is on capturing the abnormal changes, i.e., $\Delta\delta$ and in turn $\Delta P_g$, in response to the outage rather than accurate state estimations. The complete list of state estimation results for all monitored generator buses can be found in Section \ref{ch4:sec:appendix:state_estimation} of Appendix \ref{ch4:sec:appendix}.
\begin{figure}[!thpb]
\centering
\includegraphics[width=1\linewidth]{\Pic{pdf}{line_18_state_estimation_eg}}
\caption{\textit{State estimation result of the particle filter on $\delta$ and $\omega$ of Bus 33. The algorithm can estimate $\omega$ accurately, while the estimation of $\delta$ has biases after the outage. The changes in $\delta$ are sufficiently captured, which are more critical for the detection scheme.}}
\label{ch4:fig:line_18_state_estimation_eg}
\end{figure}
\begin{figure}[!thpb]
\centering
\includegraphics[width=1\linewidth]{\Pic{pdf}{detection_signal_breakdown_line_11}}
\caption{\textit{Output signals of the detection scheme for line 11 outage. Each line represents data from a bus equipped with a PMU. Abnormal disturbances in generator rather than load buses contributed to early detection in this case.}}
\label{ch4:fig:detection_signal_breakdown}
\end{figure}
One advantage of the proposed scheme is the ability to break down monitored signals and pinpoint the channels leading to a detection. Fig. \ref{ch4:fig:detection_signal_breakdown} shows such a breakdown. The upper two channels are the observable net active power information, and the lower-left one is the estimated generator information. They register different signal strength levels depending on the outage location, e.g., the magnitude of initial shock and the transient oscillation duration. The proposed scheme can detect outages as long as one of them picks up significant changes. It is clear in this case that the signals from PMU measurements do not contribute meaningfully to the detection. Instead, the changes in generated active power  on generator buses display significant abnormal fluctuations, leading to the outage detection. 


\begin{figure}[!t]
\centering
\includegraphics[width=1\linewidth]{\Pic{pdf}{line_11_monitor_statistic}}
\caption{\textit{Progression of MEWMA monitoring statistic for detecting line 11 outage. After the outage, the monitoring statistic crosses the detection threshold immediately and remains high afterward. The outage is successfully detected with no detection delay. Points are downsampled to half for clarity}}
\label{ch4:fig:line_11_monitor_statistics}
\end{figure}
The typical progression of the monitoring statistic, $T_{k}^2$, computed via MEWMA from the output signals is shown in Fig. \ref{ch4:fig:line_11_monitor_statistics}. Before the outage, the statistic remains close to zero. After the outage at the third second, it increases rapidly and crosses the threshold. Thus, the scheme raises an outage alarm, and no detection delay is incurred. More outage detection and output signal breakdown examples can be found in Section \ref{ch4:sec:appendix:detection} of Appendix \ref{ch4:sec:appendix}. In particular, the results for the scenario of no outage, missed detection, and substantial outage signal are presented. 


\subsection{Results and Discussion} 
This section presents the result computed from 1000 random simulations of each line outage. It includes the comparison with other state-of-the-art methods, the detection delay over all outage scenarios, the effect of the outage location as well as the impact of the smoothing parameter $\lambda$.

\paragraph{Detection Rate}
% Performance of the detection scheme under different lambda values in terms of detection delay for each line
\begin{figure}[!thpb]
\centering
\includegraphics[width=1\linewidth]{\Pic{pdf}{detection_rate}}
\caption{\textit{Comparison of the empirical likelihood of detection for all simulated outages under different $\lambda$s of MEWMA. While 28 out of the 35 line outages can be detected with over 90\% likelihood, larger values of $\lambda$ tend to have a higher detection rate. A small group of outages is difficult to detect regardless of the $\lambda$ value.}}
\label{ch4:fig:detection_rate}
\end{figure}
Fig. \ref{ch4:fig:detection_rate} presents the empirical likelihood of detection for all 35 simulated line outages, which is the percentage of successful detection over 1000 simulations. For both small and large values of $\lambda$, the detection scheme can detect 28 out of 35 outages over 90\% of the time. In some cases, it can be seen that larger values of $\lambda$ tend to have a better detection rate, i.e., line 8, 13, 15, and 26. The reason is that these line outages produce more severe initial shock relative to their after-outage oscillation. Hence, larger values of $\lambda$ help to capture the immediate shock. Also, a small group of outages is challenging to detect regardless of the $\lambda$ value, i.e., line 2, 6, 19, 35, and 36. Diagnostic inspection of these cases' output signals reveals that they generally produce weak system disturbances, especially from the generators, hence often not triggering an outage alarm. The weak disturbance might be explained by the fact that these lines are connected to buses that serve zero or small loads. 


\paragraph{Detection Delay}
\begin{figure}[!thpb]
\centering
\includegraphics[width=1\linewidth]{\Pic{pdf}{detection_delay_distribution}}
\caption{\textit{Comparison of the empirical distribution of detection delays in seconds for the proposed unified scheme and the scheme based on AC power flow equations. The proposed scheme has a higher percentage of zero detection delays. It can detect almost all outages within 0.2 seconds, whereas the AC detection scheme does it in 1 second.}}
\label{ch4:fig:detection_delay_distribution}
\end{figure}
The empirical distribution of the detection delays is presented in Fig. \ref{ch4:fig:detection_delay_distribution}. The figure shows the results of the proposed scheme with different $\lambda$ values and the detection scheme based on AC power flow equations from \cite{yang2020control}. Intuitively, the scheme is faster at detecting outages when the area under the curve towards the left of the figure is larger. In this case, the proposed scheme has a much higher chance of detecting outages with zero detection delay than the AC scheme. The best-performing scheme ($\lambda = 0.5$) also detects most outages within 0.2 seconds, whereas the AC scheme detects most outages within 1 second. 


\paragraph{Effect of Outage Location Relative to the PMUs}
% Overview of detection delays over different lines
\begin{figure}[!htpb]
\centering
\includegraphics[width=1\linewidth]{\Pic{pdf}{boxplot_delay_pmu}}
\caption{\textit{Box plot of the empirical distributions of detection delays in seconds for lines with at least 1 PMU nearby and those without a PMU.}}
\label{ch4:fig:boxplot_delay_pmu}
\end{figure}
In some related work and Chapter \ref{ch:detection_using_approximate_dynamics} of this thesis, significant variations of average detection delays for outages at different lines relative to the PMU locations can be observed. Fig. \ref{ch4:fig:boxplot_delay_pmu} shows a comparison of detection delays for outage lines with at least one PMU connected to it versus those with no PMU nearby. Since only ten buses are equipped with PMUs, most lines belong to the second group\footnote{Only a few lines in the second group are presented due to space constraints. All of those omitted line outages can be detected with zero mean detection delay, except for line 35, and 36, which are often undetected.}. While outages at line 11 and 19 are often detected with 0.1-second delay, most outages are detected immediately regardless of the relative position to the PMUs. Line 11 connects to the slack bus, and its outage creates a minimal disturbance in all three output channels. This result demonstrates the spatial advantage of the proposed method and its robustness to the outage locations.



\paragraph{Comparison with Other Methods}
\begin{table}[!htpb]
\caption{Detection Delay (s) Comparison of Different Detection Schemes}
\label{ch4:tab:delay_comparison_39}
\centering
\begin{tabular}{r|llll} 
\hline
\hline
\multicolumn{1}{l|}{}     & \multicolumn{4}{c}{Average Detection Delay\tablefootnote{One standard deviation appears in ().}}  \\ \hline
\multicolumn{1}{c|}{Line} & DC - full & Subspace - full  & AC &  Unified \\ \hline
2 & 1.165 (0.006)   & 2.822 (1.924) & 0.283 (0.263) & \textbf{0.012} (0.183) \\
6 & \textendash     & 3.060 (2.011) & 0.246 (0.129) & \textbf{0.052} (0.463) \\
11& \textbf{0} (0)  & 3.048 (1.969) & 0.602 (0.205) & 0.058 (0.077) \\
15& \textbf{0} (0)  & 2.634 (1.850) & 0.005 (0.034) & \textbf{0} (0) \\
19& \textendash     & 2.836 (2.018) & 0.335 (0.378) & \textbf{0.160} (0.315) \\
26& \textendash     & 2.850 (1.958)  & 0.385 (0.228) & \textbf{0} (0)      \\
\hline          
\end{tabular}
\end{table}
%% Comparison of detection delays with other methods
The proposed method's performance is also compared with three other methods in Table \ref{ch4:tab:delay_comparison_39}. The chosen outages are, in general, more difficult to detect. The first method for comparison is based on the DC power flow model from \cite{Chen2016} and the second based on subspace identification from \cite{Hosur2019}. Both of them are tested using a full PMU deployment. The third is from the method proposed in Chapter \ref{ch:detection_using_approximate_dynamics} that relies on the AC power flow model where 10 PMUs are assumed to be installed at locations used in \cite{yang2020control}. The thresholds for all methods are selected by satisfying a false alarm constraint of 1 in 30 days. The average detection delays and their standard deviations are computed from 1000 random simulations, while a dash means a missed detection. The method proposed, ``Unified'', is consistently faster at detecting outages than the other methods.


\section{Conclusion}
\label{ch4:sec:conclusion}
In this chapter, a unified framework of online transmission line outage detection is proposed. The framework utilizes information from both generator machine states and load bus algebraic variables. The signals are obtained through nonlinear state estimation of particle filters and direct measurements of PMUs. They are effectively used for outage monitoring and detection by MEWMA control charts while meeting a particular false alarm criterion. The approach is shown to be quicker at detecting outages and more robust to a priori unknown outage locations under a limited PMU deployment through an extensive simulation study. Further research can be done to improve the detection scheme’s effectiveness by investigating the optimal installation location of limited PMUs given a network of power stations. Also, it is observed that a group of lines is consistently challenging to detect regardless of the detection schemes or parameter designs used. More work needs to be done in this area so that these detection blind spots could be reduced.








% section outage_detection_using_generator_dynamics (end)