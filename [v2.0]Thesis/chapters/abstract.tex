\begin{abstract}

Power systems are critical infrastructures that modern societies depend upon. However, protecting them and ensuring a continuous supply of high-quality electricity require substantial effort. Disruptions such as transmission line outages happen frequently due to various internal and external reasons. If not detected in time, they might evolve into large-scale cascading failures. Concurrently, the increasing adoption of Phasor Measurement Unit (PMU) technology in grid modernization opens up new research opportunities for real-time power system analytics. Motivated by the need for better real-time system situational awareness and the emergence of the PMU technology, this thesis develops three novel data-driven methods to detect and identify transmission line outages.

Line outage detection aims to detect an outage as soon as possible after it happened; Identification is focused on accurately locating the disconnected line(s). There are several challenges to a successful outage detection and identification. The first is the challenge of real-time computation of fast-streaming PMU data in a large power system. The exponential number of outage line possibilities one has to consider presents another computational challenge. The second arises from the limited PMU deployment on a power system. In effect, certain parts of the system may be unobservable, leading to smaller signal-to-noise ratio. Outage of some lines may also become indistinguishable from one another. Addressing these challenges, this thesis proposes three methods for outage line detection and identification.

(1) Using nodal voltage phase angle data collected from a limited number of PMUs, a real-time dynamic outage detection scheme based on alternating current (AC) power flow model and statistical change detection theory is proposed. The method can capture system dynamics since it retains the time-variant and nonlinear nature of the power system. The method is computationally efficient and scales to large and realistic networks. Extensive simulation studies on IEEE 39-bus and 2383-bus systems demonstrated the effectiveness of the proposed method.

(2) As an extension to the first work, a unified detection framework that utilizes both generator dynamic states and nodal voltage information is proposed. The inclusion of generator dynamics makes detection faster and more robust to a priori unknown outage locations, demonstrated using the IEEE 39-bus test system. In particular, the scheme achieves an over 80\% detection rate for 80\% of the lines, and most outages are detected within 0.2 seconds. The new approach could be implemented to improve system operators' real-time situational awareness by detecting outages faster and providing a breakdown of outage signals for diagnostic purposes.

(3) On top of detection methods, a new way of identifying multiple-line outages using limited nodal voltage data is proposed. Based on the AC power flow model, voltage angle signatures of outages are extracted and used to group lines into minimal diagnosable clusters. Identification is then formulated into an underdetermined sparse regression problem solved by lasso. Tested on IEEE 39-bus system with 25\% and 50\% PMU coverage, the proposed identification method is 93\% and 80\% accurate for single- and double-line outages. This study suggests that the AC power flow is better at capturing outage patterns and sacrificing some precision could yield substantial improvement in identification accuracy. These findings could contribute to the development of future control schemes that help power systems resist and recover from outage disruptions in real time. 


\end{abstract}
