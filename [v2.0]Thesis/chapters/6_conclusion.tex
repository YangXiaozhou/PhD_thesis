\chapter{Conclusion}
\label{ch:conclusion}

Motivated by the need to improve real-time situational awareness of power system operators and the emergence of the PMU technology, this thesis sets out to develop advanced data analytic methods to detect and identify transmission line outages in real-time.

In Chapter \ref{ch:detection_using_approximate_dynamics}, a novel line outage detection scheme is proposed based on the power flow model and generalized likelihood ratio testing. The power flow model is used as the basis for predicting post-outage angle deviations. The control chart constructed based on the GLR procedure can detect any outages quickly while controlling for the false alarm rate. Notably, the proposed method can capture system dynamics since it retains the time-variant and nonlinear nature of the power system. An extensive simulation study suggests that the method performs well for outage detection, where most outages can be detected with less than one second of delay. However, longer delays are observed for outage lines with no nearby PMU sensors, prompting the need for research into a more robust detection scheme.

Chapter \ref{ch:detection_using_generator_dynamics} extends the research in the previous chapter by proposing a unified detection framework that monitors both generator dynamics and load bus dynamics. The unified framework consists of a particle filter-based nonlinear state estimator and a MEWMA-based control chart. The simulation study using the IEEE 39-bus system shows that the inclusion of generator dynamics makes detection faster and more robust to a priori unknown outage locations. In particular, 80\% of the simulated outages can be detected, and most of them are detected by 0.2 seconds after the event. 

Although encouraging results are seen from the previous two chapters, they do not address another critical and practical question regarding line outage awareness - identification. Assuming an outage detection alarm has been raised, Chapter \ref{ch:identification} proposes a novel method to accurately identify a priori unknown number of outage lines using limited PMU sensors. This work draws inspiration from sensitivity analysis of the AC power flow model and advances in underdetermined sparse regression methods. The use of lasso formulation and the LARS algorithm, in particular, overcomes the inherent combinatorial challenge of the line identification problem. Compared to the state-of-the-art in multiple-line outage identification, the proposed method achieves over 90\% and 80\% accuracy for single- and double-line outage identification. 

The guiding principle to this thesis' approach to addressing the outage detection and identification problem is deep integration between statistical monitoring and diagnostic methods and power system physical modeling. Through three chapters of work, it is demonstrated that the AC power flow model, although nonlinear and more complex than the DC version commonly used, can provide a significant advantage in both detection and identification. Still, the performance is achieved by combining the physical model with careful application of methods outside the traditional domain of power system research, e.g., statistical quality control and data mining. As PMU technology takes a central role in the modernization drive of power grids across the world, real-time outage detection and identification methods developed in this thesis can contribute to a more reliable and resilient power system. 

Three future research directions stemming from this thesis are worth pursuing: (1) The first concerns the optimal placement of limited PMUs in a network that maximizes identification accuracy. As seen in Chapter \ref{ch:identification}, the placement of PMUs influences the diagnosability of the system, suggesting the potential reward of a carefully considered placement. In general, the placement of sensors is usually formulated as a mixed-integer programming problem that is difficult to solve exactly. Also, the impact of the number of PMUs on the identification performance could be studied. (2) The second area of future research is active diagnostics for indistinguishable line outages. The problem of ambiguous outage lines has been observed in previous research and this thesis. Until now, most research efforts have focused on passively analyzing power system data to ascertain the location of outage lines. However, for the group of line outages that might be difficult to distinguish from one another, proactive identification methods might be needed. For example, targeted system inputs can be employed to specifically find out the most probably outage line among a group of highly similar candidate lines. (3) Lastly, more research can be done in the area of optimal post-disruption recovery for power systems. With contingencies like line outages detected and identified, system operators need to coordinate recovery actions to minimize the potential impact while considering various constraints. Also, uncertainties in resources available and repair duration of line outages need to be factored in when designing a recovery strategy. The ability to recover efficiently following a disruption is a critical aspect of resilient power systems. 


