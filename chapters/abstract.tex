\begin{abstract}

There is increased volatility in the power system with the addition of unconventional generation sources and loads. Power system condition monitoring is one of the critical tasks for reliable delivery of high-quality electricity. In particular, online transmission line outage detection and localization over the entire network enable timely corrective action to be taken and prevent a local event from cascading into a large scale blackout. Line outage detection aims to detect a transmission line outage as soon as possible after it happened while the localization is focused on accurately identifying the disconnected line or lines. The penetration of Phasor Measurement Unit (PMU) technology allows the collection of high-resolution time-synchronized real-time data on the network. Using voltage phase angle data collected from the PMUs, we propose a dynamic online transmission line outage detection and localization scheme developed from the full AC power flow model and statistical change detection theory. Traditional outage detection methods rely heavily on the simplified DC power flow model which largely ignores system dynamics over time. The method proposed can capture system dynamics since the time-variant and nonlinear nature of the power system are retained. The method is online friendly and scales to large networks because of the algorithm's low computational cost. Through extensive simulation study on both the detection delay and localization accuracy, the scheme is proven to be more effective than existing methods. The standard IEEE 9-bus and 39-bus test power systems are used in the simulation study. The method proposed could be incorporated into the current wide-area monitoring system and help to improve system operators' situational awareness in real time, thus improving the resilience of the power system. As an extension from this work, we would further investigate the problem by considering an economic constraint on the number of PMUs that can be installed on the power network as well as the optimal placement of available PMUs.

\end{abstract}
