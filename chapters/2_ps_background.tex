\SetPicSubDir{2_ps_background}
\SetExpSubDir{2_ps_background}

\chapter{Power System Background}
\label{ch:ps_background}
\vspace{2em}

This Chapter provides brief background information on power system modeling and simulation. In particular, relevant physical quantities and physical laws governing power systems are introduced in Section \ref{ch2:sec:ps_model}. Power system simulation and PMU data collection are introduced in Section \ref{ch2:sec:simulation}.

\section{Power System Model}
\label{ch2:sec:ps_model}

Sinusoidal physical quantities in power systems at constant frequency are phasors characterized their maximum value and phase angle. For example, voltage phasor can be represented by
\begin{equation}
    v(t) = \text{V}_{\text{max}}e^{j\theta} \,,
\end{equation}
where $\text{V}_{\text{max}}$ is the maximum value and $\theta$ is voltage phase angle. The effective value, $\text{V}$, is 
\begin{equation}
    \text{V} = \frac{\text{V}_{\text{max}}}{\sqrt{2}} \,,
\end{equation}
and voltage phasor can also be written as 
\begin{equation}
    V = \text{V}e^{j\theta} = \text{V}\angle\theta = \text{V}\cos\theta + j\text{V}\sin\theta \,,
\end{equation}
in the so-called exponential, polar, and rectangular form. The same can be written for current phasor $I$. The complex power in power systems can be obtained by the multiplying voltage and current phasor:
\begin{align}
    S &= VI^* \\
      &= \text{P} + j\text{Q} \,,
\end{align}
where $I^*$ is the complex conjugate of $I$. $\text{P}$ is real or active power and $\text{Q}$ is reactive power. 

% AC power flow model
A power system can be modeled as a network with $N$ buses of $\mathcal{N} = \{1, \dots, N\}$ connected by $L$ transmission lines of $\mathcal{L} = \{1, \dots, L\}$. The flow of real and reactive power in the network can be characterized by a set of non-linear algebraic equations called the AC power flow model. This set of equations describes the relationship between net active power injection (P), net reactive power injection (Q), voltage magnitude (V), and voltage phase angle ($\theta$) governed by Kirchhoff's circuit laws. They can be written as:
% AC power flow equations
\begin{subequations}
\label{eqn:AC_power_flow_model}
\begin{align}
\text{P}_m &= \text{V}_m \sum_{n=1}^{N} \text{V}_n \text{Y}_{mn} \cos (\theta_m - \theta_n - \alpha_{mn}) \,, \label{eqn:AC_power_flow_P}\\
\text{Q}_m &= \text{V}_m \sum_{n=1}^{N} \text{V}_n \text{Y}_{mn} \sin (\theta_m - \theta_n - \alpha_{mn}) \,, \label{eqn:AC_power_flow_Q}
\end{align}
\end{subequations}
for bus $m = 1, 2, \dots, N$ \cite{Glover2012}. Y$_{mn}$ is the magnitude of the $(m,n)_{th}$ element of the bus admittance matrix $\boldsymbol{Y}$ when the complex admittance is written in the exponential form, i.e. 
\begin{equation}
\label{eqn:admittance_expression}
\text{Y}_{mn}e^{j\alpha_{mn}} = G_{mn} + jB_{mn} \,,
\end{equation}
where $G_{mn}$ and $B_{mn}$ are the conductance and susceptance of line $\ell$ connecting bus $m$ and $n$. Elements of the bus admittance matrix corresponding to a baseline condition are usually known and this condition is also assumed for the rest of this thesis. For a large system, $\boldsymbol{Y}$ is usually a sparse matrix since any single bus only has a few incident buses, i.e., Y$_{mn} = 0$ if bus $m$ and bus $n$ are not connected. The system topology is embedded in the admittance matrix $\boldsymbol{Y}$. In particular, the admittance matrix is constructed by 
\begin{equation}
\label{eqn:admittance_matrix}
\boldsymbol{Y} = \mathbf{A [y] A}^{T}
\end{equation} where $\mathbf{A}$ is the bus to branch incidence matrix with columns representing lines and rows as buses. $\mathbf{A}^T$ is the transpose of $\mathbf{A}$. For the $l_{th}$ line transmitting power from bus $m$ to bus $n$, the $l_{th}$ column of the matrix $\mathbf{A}$ has 1 and -1 on the $m_{th}$ and $n_{th}$ position and 0 everywhere else. $\mathbf{[y]}$ is the diagonal matrix with individual line admittances on the diagonal.


\section{Power System Simulation}
\label{ch2:sec:simulation}

% Timescale for dynamic simulation

% System models for dynamic simulation

% How is outage simulated

% How is PMU data collected
For a bus equipped with PMU, V and $\theta$ are measured and available. 