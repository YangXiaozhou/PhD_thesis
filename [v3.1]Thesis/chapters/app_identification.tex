\SetPicSubDir{5_identification}

\chapter{PMU Placement and Additional Identification Results} % (fold)
\label{ch5:sec:appendix}

\section{Genetic Algorithm-generated PMU Placement}
\label{ch5:sec:optimal_placement}

The pairwise correlation heatmap of the signature map based on the genetic algorithm-generated optimal PMU placement is shown in Fig. \ref{ch5:fig:best_correlation_heatmap}. It can be observed that the number of lines with high pairwise correlation has significantly reduced compared to that of a random PMU placement shown in Fig. \ref{ch5:fig:correlation_heatmap}.
\begin{figure}[!h]
\centering
\includegraphics[width=1\linewidth]{\Pic{pdf}{best_correlation_heatmap}}
\caption{\textit{Heatmap of pairwise correlation between columns of the signature map constructed from an optimal placement of 19 PMUs on the 39-bus system. The placement is generated by minimizing the average mutual coherence of the signature map using a genetic algorithm. Only correlations higher than 0.9 are plotted.}}
\label{ch5:fig:best_correlation_heatmap}
\end{figure}


\section{Additional Identification Results} % (fold)
\label{ch5:sec:additional_identification_results}

Additional results of the proposed method's identification performance are reported here. 

\subsection{Average Performance}

The average identification performance of the proposed method and other methods under comparison for single-line outages with 75\% PMU coverage is shown in Fig. \ref{ch5:fig:single_outage_average_29pmus}. 
% Single outage with 75% coverage
\begin{figure}[!th]
\centering
\includegraphics[width=1\linewidth]{\Pic{pdf}{single_outage_average_29pmus}}
\caption{\textit{Box-plots of single-line outage identification results for DC-based, correlation-based, and the proposed method. Results are based on 200 random simulation runs under a 75\% PMU coverage in the New England 39-bus system. Each method has two sets of results: accuracy of the original identification and of that augmented with MDCs.}}
\label{ch5:fig:single_outage_average_29pmus}
\end{figure}

The average identification performance of the proposed method and other methods under comparison for double-line outages with 25\% PMU coverage is shown in Fig. \ref{ch5:fig:double_outage_average_10pmus} and 75\% PMU coverage in Fig. \ref{ch5:fig:double_outage_average_29pmus}.
% Double outage with 25% coverage
\begin{figure}[!th]
\centering
\includegraphics[width=1\linewidth]{\Pic{pdf}{double_outage_average_10pmus}}
\caption{\textit{Box-plots of double-line outage identification results for DC-based, correlation-based, and the proposed method. ``All correct'' (top) and ``half correct'' (bottom) results are based on 200 random simulation runs under a 25\% PMU coverage in the New England 39-bus system. Each method has two sets of results: accuracy of the original identification and of that augmented with MDCs.}}
\label{ch5:fig:double_outage_average_10pmus}
\end{figure}
% Double outage with 75% coverage
\begin{figure}[!th]
\centering
\includegraphics[width=1\linewidth]{\Pic{pdf}{double_outage_average_29pmus}}
\caption{\textit{Box-plots of double-line outage identification results for DC-based, correlation-based, and the proposed method. ``All correct'' (top) and ``half correct'' (bottom) results are based on 200 random simulation runs under a 75\% PMU coverage in the New England 39-bus system. Each method has two sets of results: accuracy of the original identification and of that augmented with MDCs.}}
\label{ch5:fig:double_outage_average_29pmus}
\end{figure}

\clearpage
\subsection{Effect of Minimal Diagnosable Cluster}

The impact of the MDC correlation threshold on the outage identification accuracy and the proportion of single-element MDCs is reported in Table \ref{ch5:tab:impact_mdc_threshold_10pmus} for 25\% PMU coverage and in Table \ref{ch5:tab:impact_mdc_threshold_29pmus} for 75\% PMU coverage. 
\begin{table}[!th]
\caption{Impact of Minimal Diagnosable Cluster Threshold on Identification Precision-Accuracy Trade-off Using Lasso+MDC Under 25\% PMU Coverage}
\label{ch5:tab:impact_mdc_threshold_10pmus}
\centering
\begin{tabular}{llll}
\hline
\hline
Threshold ($\rho^*$)  & Single-element MDC (\%) & Single-line & Double-line \\
\hline
0.80 & 0.12 (0.04) & 0.88 (0.09) & 0.54 (0.09) \\
0.84 & 0.17 (0.05) & 0.89 (0.08) & 0.50 (0.09) \\
0.88 & 0.23 (0.05) & 0.87 (0.08) & 0.46 (0.09) \\
0.93 & 0.31 (0.06) & 0.85 (0.08) & 0.41 (0.09) \\
0.95 & 0.35 (0.06) & 0.86 (0.09) & 0.41 (0.11) \\
0.98 & 0.39 (0.06) & 0.84 (0.09) & 0.38 (0.11) \\
0.99 & 0.49 (0.07) & 0.78 (0.09) & 0.29 (0.10) \\
\hline 
\end{tabular}
\end{table}

\begin{table}[!th]
\caption{Impact of Minimal Diagnosable Cluster Threshold on Identification Precision-Accuracy Trade-off Using Lasso+MDC Under 75\% PMU Coverage}
\label{ch5:tab:impact_mdc_threshold_29pmus}
\centering
\begin{tabular}{llll}
\hline
\hline
Threshold ($\rho^*$)  & Single-element MDC (\%) & Single-line & Double-line \\
\hline
0.80 & 0.57 (0.05) & 0.98 (0.03) & 0.79 (0.06) \\
0.84 & 0.62 (0.06) & 0.99 (0.03) & 0.79 (0.03) \\
0.88 & 0.71 (0.06) & 0.98 (0.03) & 0.79 (0.04) \\
0.93 & 0.77 (0.06) & 0.98 (0.03) & 0.79 (0.05) \\
0.95 & 0.78 (0.06) & 0.97 (0.03) & 0.79 (0.04) \\
0.98 & 0.84 (0.06) & 0.96 (0.02) & 0.78 (0.04) \\
0.99 & 0.86 (0.06) & 0.95 (0.03) & 0.76 (0.05) \\
\hline 
\end{tabular}
\end{table}

\newpage
\subsection{Effect of Measurement Noise}

The performance of the proposed method under different measurement noise levels is reported in Fig. \ref{ch5:fig:impact_noise_10pmus} for 25\% PMU coverage and in Fig. \ref{ch5:fig:impact_noise_29pmus} for 75\% PMU coverage. 
\begin{figure}[!th]
\centering
\includegraphics[width=1\linewidth]{\Pic{pdf}{impact_noise_10pmus}}
\caption{\textit{Impact of measurement noise on identification performance of the proposed method under a 25\% PMU coverage. Performance using data with noise standard deviation varying from 0\% to 10\% of $|\Delta\boldsymbol{\theta}|$ is reported by median accuracy of single- and double-line outages using Lasso and Lasso+MDC. }}
\label{ch5:fig:impact_noise_10pmus}
\end{figure}

\begin{figure}[!th]
\centering
\includegraphics[width=1\linewidth]{\Pic{pdf}{impact_noise_29pmus}}
\caption{\textit{Impact of measurement noise on identification performance of the proposed method under a 75\% PMU coverage. Performance using data with noise standard deviation varying from 0\% to 10\% of $|\Delta\boldsymbol{\theta}|$ is reported by median accuracy of single- and double-line outages using Lasso and Lasso+MDC. }}
\label{ch5:fig:impact_noise_29pmus}
\end{figure}
